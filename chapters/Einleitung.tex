\documentclass[../Bachelor_LennartKoliwer.tex]{subfiles}
\graphicspath{{\subfix{../images/}}}
\begin{document}

% Diese Arbeit beschäftigt sich mit Dynamic Mode Decomposition (DMD). DMD ist ein Algorithmus, mit dem man versucht, in einem dynamischen System mit Hilfe von Messdaten, Zustände vorherzusagen. Da es verschiedene Varianten von DMD gibt beschränken wir uns auf Exact DMD. Im Gegensatz zum ursprünglichen Algorithmus, der eine Reihe an Zeitpunkten fordert, braucht man bei Exact DMD nur Zeitpunkt Paare, die unabhängig voneinander gemessen werden können. Nehmen wir mal das Beispiel von Wasserströmungen. Da müsste man innerhalb eines beschränkten Raumes an diskreten Punkten Messungen durchführen. Was daraus resultiert sind riesige mit Datenpunkten gefüllte Matrizen. Was man also versucht ist sich die Singulärwerte dieser Matrix anzugucken, da diese aussagen, wohin und wie stark sich der Fluss bewegt. Wenn man versuchen möchte, mit allen Daten das System zu simulieren, wird man sehr schnell merken, dass der Rechenaufwand gewaltig werden kann. Eine Möglichkeit den Rechenaufwand zu verringern ist die Matrix verkleinern, indem man alle Singulärwerte entfernt, die $0$ sind. Man könnte aber noch einen Schritt weitergehen und sogar die Singulärwerte entfernen, die einfach zu klein sind, um einen merkbaren Einfluss zu haben. Damit beschäftigt sich das Paper von M. Gavish und D. L. Donoho \cite{OHT}, was in dieser Arbeit näher beleuchtet wird.

Motivation DMD 
Daten haben fehler
dmd mag fehler nicht  





!
%Da es verschiedene Varianten von DMD gibt beschränken wir uns auf Exact DMD. Im Gegensatz zum ursprünglichen Algorithmus, der eine Reihe an Messzeitpunkten fordert, welche einen einheitlichen Abstand voneinander haben, benutzt man bei Exact DMD Daten von verschiedenen Messdurchführungen.
lalala
\begin{align}
	A=\begin{pmatrix}
		  a_{11} & \cdots & a_{1m} \\
		  \vdots & \ddots & \vdots \\
		  a_{n1} & \cdots & a_{nm}
	  \end{pmatrix}
\end{align}
lalala

\end{document}