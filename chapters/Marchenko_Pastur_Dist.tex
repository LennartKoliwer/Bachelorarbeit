\documentclass[../Bachelor_LennartKoliwer.tex]{subfiles}
\graphicspath{{\subfix{../images/}}}

\begin{document}

\begin{lemma} \label{lem:betaUmformung}
	Sei $\beta\in(0,1]$ und $\beta_\pm=(1\pm\sqrt{\beta})^2$. Dann gilt
	\begin{align}
		\beta_+-\beta_- & =4\sqrt{\beta}  \\
		\beta_++\beta_- & =2\beta+2       \\
		\beta_+\beta_-  & =(\beta-1)^2    \\
		m               & = \beta+1       \\
		r               & = 4\sqrt{\beta}
	\end{align}
\end{lemma}

Beweis, dass die Lösung der Marchenko Pastur Verteilung mit $\beta_-\leq \mu_\beta \leq\beta_+$ und $\beta_\pm=(1\pm\sqrt{\beta})^2$ also
\begin{align}
	\int_{\beta_-}^{\mu_\beta} \frac{\sqrt{(\beta_+-t)(t-\beta_-)}}{2\pi \beta t} dt = \frac12
\end{align}
ergeben kann.

\textcolor{red}{Umgangssprache ist überall :(. Definitionen und Sätze aufschreiben. Und aus den align Umgebungen vielleicht noch die Label auf ein angenehmes Minimum reduzieren.}
\begin{proof}
	Da das Integral mit $\mu_\beta=\beta_-$ gleich $0$ ist und das Integral stetig innerhalb des Intervalls $[\beta_-,\beta_+]$ ist, müssen wir nur noch zeigen, dass das Integral bei $\mu_\beta=\beta_+$ gleich $1$ ist, damit wir den Zwischenwertsatz nutzen können.
	Dafür bietet sich an, das Integral
	\begin{align}
		\frac{1}{2\pi \beta}\int_{\beta_-}^{\beta_+} \frac{\sqrt{(\beta_+-t)(t-\beta_-)}}{t} dt\
	\end{align}
	mit $\mu_\beta=\beta_+$ auszurechnen. Wir beginnen damit eine Substitution zu finden.
	Dafür bauen wir uns also einen Weg $\delta$, der von $\beta_-$ zu $\beta_+$ läuft.
	Sei $m=\frac{\beta_++\beta_-}{2}$ der Mittelpunkt zwischen $\beta_-$ und $\beta_+$ und $r=\frac{\beta_+-\beta_-}{2}$ der Abstand von Mittelpunkt zu den Grenzen, dann brauchen wir nur noch eine Funktion, die von $-1$ bis $1$ läuft, damit wir den Weg von einer Grenze zur anderen ablaufen. Dafür eignet sich der Cosinus von $-\pi$ bis $0$. Da dieser aber eine gerade Funktion ist, können wir auch von $0$ bis $\pi$ gehen.

	Damit erhalten wir
	\begin{align}
		\delta\colon [0,\pi] & \to \RR                     \\
		\varphi              & \mapsto m+r\cos(\varphi)\,.
	\end{align}
	Da $\delta$ injektiv und stetig differenzierbar ist, substituieren wir $t$ mit $\delta$ und erhalten
	\begin{align*}
		  & -\frac{1}{2\pi \beta}\int_{\pi}^{0} \frac{\sqrt{\left(\beta_+-(m+r\cos(\varphi))\right)\left(\left(m+r\cos(\varphi)\right)-\beta_-\right)}}{m+r\cos(\varphi)} r\sin(\varphi)d\varphi
		\intertext{
			setzen wir \(\beta_+=m+r\) und \(\beta_-=m-r\)
		}
		= & -\frac{1}{2\pi \beta}\int_{\pi}^{0} \frac{\sqrt{\left((m+r)-(m+r\cos(\varphi))\right)\left(\left(m+r\cos(\varphi)\right)-(m-r)\right)}}{m+r\cos(\varphi)} r\sin(\varphi)d\varphi
		\\[0.3em]
		= & -\frac{1}{2\pi \beta}\int_{\pi}^{0} \frac{\sqrt{\left(r-r\cos(\varphi)\right)\left(r+r\cos(\varphi)\right)}}{m+r\cos(\varphi)} r\sin(\varphi)d\varphi
		\\[0.3em]
		= & -\frac{1}{2\pi \beta}\int_{\pi}^{0} \frac{\sqrt{\left(r^2-r^2\cos^2(\varphi)\right)}}{m+r\cos(\varphi)} r\sin(\varphi)d\varphi
		\\[0.3em]
		= & -\frac{1}{2\pi \beta}\int_{\pi}^{0} \frac{\left(r-r\cos(\varphi)\right)}{m+r\cos(\varphi)} r\sin(\varphi)d\varphi
		\\[0.3em]
		= & -\frac{1}{2\pi \beta}\int_{\pi}^{0} \frac{r^2\sin^2(\varphi)}{m+r\cos(\varphi)} d\varphi
		\intertext{Nutzen wir hier Lemma \ref{lem:betaUmformung}, erhalten wir}
		= & -\frac{1}{2\pi \beta}\int_{\pi}^{0} \frac{4\beta\sin^2(\varphi)}{\beta+1+2\sqrt{\beta}\cos(\varphi)} d\varphi
		\\[0.3em]
		= & \frac{1}{2\pi \beta}\int_{0}^{\pi} \frac{4\beta\sin^2(\varphi)}{\beta+1+2\sqrt{\beta}\cos(\varphi)} d\varphi\,.
	\end{align*}
	Ab hier müssen wir eine Fallunterscheidung vornehmen.

	Fall 1: $\beta=1$.
	\begin{align}
		  & \frac{1}{2\pi}\int_{0}^{\pi} \frac{4\sin^2(\varphi)}{2+2\cos(\varphi)} d\varphi                                           \\
		= & \frac{1}{2\pi}\int_{0}^{\pi} \frac{2^2-2^2\cos^2(\varphi)}{2+2\cos(\varphi)} d\varphi                                     \\
		= & \frac{1}{2\pi}\int_{0}^{\pi} \frac{\left(2-2\cos(\varphi)\right)\left(2+2\cos(\varphi)\right)}{2+2\cos(\varphi)} d\varphi \\
		= & \frac{1}{\pi}\int_{0}^{\pi} 1-\cos\left(\varphi\right) d\varphi                                                           \\
		= & 1\,.
	\end{align}

	Fall 2: $\beta\in(0,1)$.
	\textcolor{red}{hier noch eine Motivation finden den Residuensatz nutzen zu wollen}
	Bei genauerer Betrachtung des Integranten, nennen wir ihn $f$, kann man erkennen, dass man ihn in zwei Teile separieren kann, genau
	\begin{align}
		f(\varphi)=a\cdot\sin^2(\varphi)\quad\text{mit}\quad a=\frac{1}{m+r\cos\left(\varphi\right)}\,.
	\end{align}

	Da der Cosinus $2\pi$-periodisch ist  und $m$ und $r$ konstant sind, ist auch $a$ $2\pi$-periodisch undspiegelt sich an $\varphi=\pi$. $\sin^2(\varphi)$ hingegen ist $\pi$-periodisch. Daraus erkennen wir in $f$ eine $2\pi$-Periodizität mit einer Symmetrie um $\pi$. Wir können das integral also von $0$ bis $2\pi$ laufen lassen und dann halbieren.
	Wir haben also jetzt
	\begin{align}
		\frac{1}{\pi} & \int_{0}^{2\pi} \frac{\sin^2\left(\varphi\right)}{\beta+1+2\sqrt{\beta}\cos\left(\varphi\right)} d\varphi
	\end{align}
	Sei nun $z=Re^{i\varphi}$, $\cos\left(\varphi\right) = \frac{e^{i\varphi}+e^{-i\varphi}}{2}$ und $\sin\left(\varphi\right) = \frac{e^{i\varphi}-e^{-i\varphi}}{2i}$ und setzen ein
	\begin{align}
		-\frac{1}{4\pi}\int_{0}^{2\pi}\frac{\left(e^{i\varphi}-e^{-i\varphi}\right)^2}{\left(e^{i\varphi}+\sqrt{\beta}\right)\left(e^{-i\varphi}+\sqrt{\beta}\right)} d\varphi\,.
	\end{align}
	Um den Residuensatz anwenden zu wollen brauchen wir ein Wegintegral, das wir uns bauen können, indem wir eine Definition für komplexe Wegintegrale nutzen (Definition 3.7 Funktheo Skript).
	Durch Multiplizieren mit einer passenden $1$  können wir also aus
	\begin{align}
		  & -\frac{1}{4\pi}\int_{0}^{2\pi}\frac{\left(e^{i\varphi}-e^{-i\varphi}\right)^2}{\left(e^{i\varphi}+\sqrt{\beta}\right)\left(e^{-i\varphi}+\sqrt{\beta}\right)} \cdot \frac{ie^{i\varphi}}{ie^{i\varphi}}d\varphi \\
		= & -\frac{1}{4\pi i}\int_{|z|=1} \frac{\left(z-z^{-1}\right)^2}{z\left(z+\sqrt{\beta}\right)\left(z^{-1}+\sqrt{\beta}\right)} dz                                                                                   \\
		= & -\frac{1}{4\pi i}\int_{|z|=1} \frac{z^2-2+z^{-2}}{\left(z+\sqrt{\beta}\right)\left(z\sqrt{\beta}+1\right)} dz
		\intertext{machen. Um die $z^{-2}$ in Zähler zu eliminieren multiplizieren wir erneut mit einer passenden $1$}
		= & -\frac{1}{4\pi i}\int_{|z|=1} \frac{z^4-2z^2+1}{z^2\left(z+\sqrt{\beta}\right)\left(z\sqrt{\beta}+1\right)} dz\,.
	\end{align}
	Sichtbar werden hier die Singularitäten $z=0$, welche eine Polstelle 2. Ordnung ist und $z=-\sqrt{\beta}$, welche eine einfache Polstelle ist. Die Singularität $-\sqrt{\beta}^{-1}$ liegt außerhalb des Einheitskreises und damit $\ind\left(\gamma,-\sqrt{\beta}^{-1}\right)=0$. Weiter nennen wir den Weg des Einheitskreises $\gamma$.
	Der Residuensatz gibt uns dann
	\begin{align}
		= & -\frac{1}{4\pi i}\int_{\gamma} \frac{z^4-2z^2+1}{z^2\left(z+\sqrt{\beta}\right)\left(z\sqrt{\beta}+1\right)} dz \\
		= & -\frac{1}{4\pi i}2\pi i\sum_{z\in\CC}\ind(\gamma,z)\res(f,z)
		\intertext{Für unsere beiden Singularitäten bekommen wir also}
		= & -\frac{1}{2}\left(\res(f,-\sqrt{\beta})+\res(f,0)\right)\,.
	\end{align}
	Fangen wir mit $z=-\sqrt{\beta}$ an.
	\begin{align}
		  & \res(f,-\sqrt\beta)=\lim_{z\to-\sqrt{\beta}}\left(\left(z+\sqrt{\beta}\right)f(z)\right) \\
		= & \lim_{z\to-\sqrt{\beta}} \frac{z^4-2z^2+1}{z^2\left(z\sqrt{\beta}+1\right)}              \\
		= & \frac{1-\beta}{\beta}
	\end{align}
	Weiter mit $z=0$.
	\begin{align}
		  & \res(f,0)=\lim_{z\to0}\frac{d}{dz}\left(z^2f(z)\right)
		=\lim_{z\to0} \frac{d}{dz} \frac{z^4-2z^2+1}{\left(z+\sqrt{\beta}\right)\left(z\sqrt{\beta}+1\right)}                                                                                                                                 \\[0.3em]
		= & \lim_{z\to0}\frac{\left(4z^3-4z\right)\left(z+\sqrt{\beta}\right)\left(z\sqrt{\beta}+1\right)-\left(z^4-2z^2+1\right)\left(2z\sqrt{\beta}+1+\beta\right)}{\left(\left(z+\sqrt{\beta}\right)\left(z\sqrt{\beta}+1\right)\right)^2} \\
		= & -\frac{1+\beta}{\beta}\,.
	\end{align}
	Zusammengefasst erhalten wir
	\begin{align}
		= & -\frac{1}{2}\left(\res(f,-\sqrt{\beta})+\res(f,0)\right)             \\
		= & -\frac{1}{2}\left(\frac{1-\beta}{\beta}-\frac{1+\beta}{\beta}\right) \\
		= & 1\,.
	\end{align}
	Juhu
\end{proof}


\begin{theorem}[Zwischenwertsatz \cite{Ana1}\textcolor{red}{weiß noch nicht ob das rein sollte oder nicht weil zu trivial vielleicht}]
	Sei $f\colon [a,b]\to \RR$ eine stetige Funktion und $n\in\RR$. Weiter sei $f(a)<n$ und $f(b)>n$. Dann gibt es ein $c\in[a,b]$, sodass $f(c)=n$.
\end{theorem}



\end{document}